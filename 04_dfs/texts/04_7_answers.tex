\cleardoublepage
\header{Ответы}
\answer{provetree} То, что не будет циклов, видимо, можно доказывать многими способами. Приведу идею
самого простого из пришедших мне сейчас в голову доказательств. Пронумеруем все вершины в том порядке, 
в котором мы их находили. В дереве поиска в глубину из каждой вершины $u$ выходим несколько (ноль или больше)
рёбер в её "<сыновья"> "--- вершины, которые мы нашли из $u$, значит, их номера больше, чем у $u$, "---
а также ровно одно ребро в "<предка"> "--- вершину, \textit{из} которой мы нашли $u$ (такие ребра есть
у всех вершин, кроме корня) "--- номер этой вершины меньше, чем у $u$. Пусть есть цикл. Рассмотрим в нем вершину с 
наибольшим номером. В неё входят \textit{два} ребра, принадлежащие этому циклу, и потому идущие и вершин с \textit{меньшими}
номерами, чем у нашей. Противоречие.

В случае связного графа это всегда будет остовное дерево графа, т.е. покрывающее все вершины. Но в случае
несвязного графа это будет или остовное дерево одной компоненты связности "--- если запускаем просто $find(i)$,
"--- или остовный лес, покрывающий все вершины "--- если запускаем вторым из перечисленных в п. \ref{howtocall}
способом.

\answer{fillwas}{Ясно, что, если в текущей компоненте связности некоторые элементы массива $was$ изначально были 
ненулевыми, то в эти вершины мы не войдём. Скорее всего, это немного не то, что мы хотели (хотя, конечно,
все зависит от задачи. Могут быть задачи, хотя я таких с ходу не припомню, в которых нам именно это и надо).}

\answer{ambigousbi}{Да, конечно, количество решений равно $2^k$, где $k$ "--- количество компонент связности графа.
В пределах одной компоненты есть два способа раскраски, отличающиеся инвертацией всех вершин. 

Вообще, есть элементарная неоднозначность: можно инвертировать все вершины сразу и получить новое решение "---
значит, решение \textit{всегда} неоднозначно. Но даже если решения, отличающиеся инвертацией \textit{всех} вершин, 
считать одинаковыми, то все равно в несвязных графах решение неоднозначно.}

\answer{whichtask}{Конечно, на вопрос "<является ли данный граф двудольным">.}

\answer{BFS:bipartie}{Ну что"=нибудь в следующем стиле (конечно, поиск в ширину я реализую очередью)}
\begin{codesample}\begin{verbatim}
var q:array[1..n] of integer;
    was:array[1..n] of integer;
    l,r:integer;
    cur:integer;
    j:integer;
...
fillchar(was,sizeof(was),0);
l:=1;r:=1;
q[1]:=1;
was[1]:=1;
while l<=r do begin
      cur:=q[l];
      inc(l);
      for j:=1 to n do
          if (gr[cur,j]<>0)and(was[j]=0) then begin
             was[j]:=3-was[cur];
             inc(r);
             q[r]:=j;
          end;
end;
\end{verbatim}
\end{codesample}
Массив $q$ "--- очередь, массив $was$ "--- номера доли. Это работает для связного графа, в противном случае ещё
нужен внешний цикл с проверкой $was$. Надеюсь, тут ошибок немного.

\answer{tripartie}{Ну понятно, почему :) Для двудольности, покрасив одну вершину, мы тут же знаем, как красить
соседние с ней, т.к. есть всего два варианта, а один из них уже занят. В трехдольности так не получится.}

\answer{checkiftree}{Собственно, в подсказке я уже сказал, как надо все делать. Осталось привести пример программы.}
\begin{codesampleo}\begin{verbatim}
procedure find(i,p:integer);
begin
if was[i]<>0 then begin
   не дерево!
   exit;
end;
was[i]:=1;
for i:=1 to n do
    if (gr[i,j]=1)and(j<>p) then
       find(j,i);
end;
\end{verbatim}
\end{codesampleo}
Дополнительный параметр $p$ здесь "--- номер вершины"=предка. Вызываем эту процедуру из главной программы, 
конечно, передавая в качестве 
вершины"=предка номер несуществующей вершины, например, ноль, если нумеруем вершины с единицы.

Да, ещё не забудьте, что для проверки, является ли граф деревом, надо запустить $find(1)$ и проверить, что вы побывали
во всех вершинах, а для проверки, является ли граф лесом, надо пробежаться по всем вершинам и запускать $find$
оттуда, где ещё не бывали.

\answer{Eulercriteria}{Окончательный критерий "--- если в графе степени всех вершин чётны плюс все компоненты связности,
кроме, может быть, одной, состоят из одной вершины (т.е. это связный граф и ещё несколько изолированных вершин).}

\answer{directEuler}{Ну что тут писать-то? Все сказано в абзаце перед задачей.}

\answer{directEulerpath}{Критерий такой: ровно одна вершина с исходящей степенью на единицу больше входящей,
и ровно одна "--- со входящей, на единицу большей исходящей; у остальных эти степени должны быть равны. 
Ну и обычные условия на связность. Пишется так же, как и эйлеров цикл в орграфе, только сначала надо найти ту самую вершину,
где входящая на единицу больше исходящей (именно так, т.к. мы пойдём по инвертированным рёбрам!)}

\answer{acyclicnontree}{Ясно, что могут быть ациклические графы "--- не деревья. Например, три вершины и 
ребра $1\to 2$, $1\to 3$ и $2\to 3$. Если снимем ориентацию с рёбер, то могут появиться циклы,
которых раньше не было из"=за того, что ребра были ориентированы.}

\answer{Knuthalgorithm}{Приведу код, только сначала несколько комментариев про хранение графа списком смежных
вершин. Буду использовать настоящие списки, т.е.%
\footnote{Замечу, что это очень синтаксически странная конструкция: я использую идентификатор $pv$ до того,
как объяснил, что он значит. Паскаль такое допускает при выполнении двух условий: во"=первых, все должно быть в одной 
"<секции"> \texttt{type}, во"=вторых, должен быть определённый порядок: то ли сначала определён $tv$, потом $pv$,
то ли наоборот, я сейчас точно не помню. Если этот код не компилится, поменяйте их местами.}}
\begin{codesample}\begin{verbatim}
type tv=record
          v:integer;
          next:pv;
       end;
     pv=^tv;
var gr:array[1..maxN] of pv;
\end{verbatim}
\end{codesample}
Здесь $tv$ "--- очередной элемент списка, хранящий одно ребро (т.е. одну смежную вершину):. $v$ "---
номер этой вершины, $pv$ "--- указатель на следующее ребро (на следующий элемент типа $tv$), или $nil$,
если такого ребра нет.
$gr$ хранит граф: для каждой вершины "--- указатель на первое ребро, \textit{в}ходящее в эту вершины (или $nil$,
если таких рёбер нет).

В данной задаче мы будем хранить только входящие ребра, т.к. исходящие нам не нужны (я говорил об этом
в подсказке). В других случаях для ориентированного графа могут понадобиться два массива отдельно для
входящих и исходящих рёбер; для неориентированного графа, конечно, нужен один массив.

Алгоритм:
\begin{codesamplec}{3}\begin{verbatim}
...
var st:array[1..maxN] of integer;
    nst:integer;
    d:array[1..maxN] of integer;
    u,v:integer;
    n,m:integer;
    nv:pv;
    ans:array[1..maxN] of integer;
    pos:integer;
    
begin
...
fllchar(gr,sizeof(gr),0);
fillchat(d,sizeof(d),0);
read(f,n,m);
for i:=1 to m do begin
    read(f,u,v);
    new(nv);
    nv^.v:=u;
    nv^.next:=gr[v];
    gr[v]:=nv;
    inc(d[u]);
end;
nst:=0;
for i:=1 to n do
    if d[i]=0 then begin
       inc(nst);
       st[nst]:=i;
    end;
pos:=n;
for i:=1 to n do begin
    {должно быть nst>0}
    v:=st[nst];
    dec(nst);
    ans[pos]:=v;
    dec(pos);
    nv:=gr[v];
    while nv<>nil do begin
          dec(d[nv^.v]);
          if d[nv^.v]=0 then begin
             inc(nst);
             st[nst]:=nv^.v;
          end;
          nv:=nv^.next;
    end;
end;
\end{verbatim}
\end{codesamplec}
$st$ "--- массив (стек) стоков; $nst$ "--- количество элементов в нем (т.е. количество стоков в текущем графе).
$d$ "--- массив исходящих степеней (т.е. $d[i]$ "--- исходящая степень $i$"=ой вершины). $ans$ "--- массив"=ответ,
$pos$ "--- позиция в этом массиве, куда мы должны будем поставить очередную вершину.

Сначала считываем граф. Я специально привожу этот текст, чтобы вы видели, как хранить граф списком смежных вершин.
Считаем, что граф задан списком рёбер: т.е. во входном файле сначала количества вершин ($n$) и рёбер ($m$), а потом 
по два числа на строке, задающие две вершины "--- откуда и куда идёт ребро. Поэтому считываем сначала эти количества,
а потом сами ребра. Каждое ребро $u\to v$ надо добавить в список рёбер, входящих в вершину $v$, т.е. в список $gr[v]$.
Посмотрите, как это делается. Тут небольшая путаница с тем, что ребро идёт из вершины $u$, поэтому приходится
писать $nv.v:=u$, но это мелочи. Может быть, можно было придумать более хорошие имена полям и переменным. Обратите
внимание, что, как всегда при вставке в список, мы вставляем в его начало, а не в конец. Заодно параллельно
считаем в массиве $d$ исходящие степени.

После этого формируем начальный список стоков $st$, пробегаясь по массиву $d$ и ища там нули. 

Далее основная часть. Мы должны $n$ раз подряд взять сток, поставить его в выходной массив и удалить его из графа.
Каждый раз сток точно найдётся, т.к. граф ациклический, поэтому все время должно быть $nst>0$. Берём очередной сток
(конечно, последний из массива $st$ "--- его проще удалять, чем первый), удаляем его из массива $st$ (командой
$dec(nst)$ просто), ставим в выходной массив и пробегаемся по входящим рёбрам, обратите внимание как. Для каждого
ребра просто уменьшаем на единицу исходящую степень соответствующей вершины и, если она стала стоком, заносим её
в массив $st$. Частая ошибка здесь "--- забыть написать \verb'nv:=nv^.next', чтобы перейти к следующему ребру. Это вам не
for, который переменную цикла автоматически увеличивает.

\answer{connecteddirect}{Например, граф с двумя вершинами и одним ребром $2\to 1$ связен, но при запуске
поиска в глубину $find(1)$ во вторую вершину мы не попадём.}

\answer{acyclicrepeat}{Три вершины, три ребра: $1\to 2$, $1\to 3$, $2\to 3$: запустившись $find(1)$,
мы два раза попробуем попасть в третью вершину.}

\answer{Knuthforacycliccheck}{Напрямую заменить нельзя, т.к. алгоритм Кнута не всегда даст какую"=то последовательность
вершин. Но наоборот: он даст какую"=то последовательность вершин тогда и только тогда, когда граф ациклический.
Т.е. приспособить алгоритм Кнута можно следующим образом: запускаем его и, если он нормально завершается, то
граф ациклический, иначе нет. А что значит нормально завершается? Единственное, что ему может помешать "---
может оказаться, что в очередной момент $nst=0$, т.е. в текущем графе нет стоков. Несложно понять, что это
будет тогда и только тогда, когда граф не ациклический. Таким образом, может в алгоритм Кнута добавить одну проверку
внутрь цикла и получить алгоритм проверки графа на ацикличность (а массив $ans$ тогда, конечно, не нужен будет).}

\answer{transitive}{Рассмотрим тот граф, который приведён в подсказке: три вершины, два ребра: $1\to 2$ и $3\to 2$.
В соответствии с нашим определением "<компонент слабой связности"> вершины 1 и 2 должны лежать в одной компоненте,
2 и 3 тоже, а 1 и 3 нет (т.к. ни от 1 до 3, ни от 3 до 1 добраться нельзя). Поэтом такое определение
бессмысленно в том смысле, что вершины не получается разбить на компоненты слабой связности. Ясно, что проблема
именно в том, что нарушается требование транзитивности%
\footnote{На самом деле, пусть про некоторые пары вершин (или вообще любых объектов), сказано, что эти пары "<хорошие">.
Тогда, чтобы вершины можно было разбить на "<компоненты хорошести">, т.е. на группы такие, что
в пределах каждой группы все пары хорошие, а между группами "--- нет, необходимо и достаточно
выполнения трёх условия: рефлексивности (что каждая вершина сама с собой образует хорошую пару),
симметричности (что если $u$ и $v$ хорошая пара, то и $v$ и $u$ тоже), и транзитивности
(если $u$ и $v$ хорошая пара, и $v$ и $w$ тоже, то и $u$ и $w$ тоже).}.}

\answer{badSCC}{Пример графа, для которого такой алгоритм не работает: три вершины, ребра $1\to 3$, $3\to 1$ и $1\to 2$.
Если запустим первый поиск в глубину из вершины 1, то результат "<топсорта"> будет именно порядок 1, 2, 3, и,
пойдя в неинвертированном графе справа налево, запустившись первым же запуском $find(3)$, мы посетим все три
вершины, что неправильно. Как "<на пальцах"> объяснить, чем таким этот алгоритм отличается от верного, я не знаю.}

\answer{condensationisacyclic}{Пусть в конденсации есть цикл. Но тогда возьмём две вершины этого цикла "---
пусть это вершины 1 и 2. В конденсации по этому циклу можно дойти и из 1 в 2, и из 2 в 1. Тогда возьмём в изначальном графе 
две вершины $1'$ и $2'$ из компонент сильной связности, соответствующих вершинам 1 и 2 конденсации. Несложно показать,
что тогда и из $1'$ в $2'$ и в обратную сторону можно дойти в начальном графе, что противоречит тому,
что они лежат в разных компонентах сильной связности.}

\answer{howtosort}{Несложно видеть, что топсорт как раз и сортирует вершины по времени выхода, только
в порядке убывания. Он ведь на последнее место в выходном массиве ставит вершину, из которой мы вышли первой, 
и т.д. Поэтому сортировать вершины по времени выхода надо аналогично топсорту. По времени входа сортировать
тоже аналогично, только ставить вершину в выходной массив надо в начале процедуры $find$. Это чем-то аналогично
сортировке подсчётом.}

\answer{bridgeone}{Для мостов можно. Для точек сочленения нет, т.к. может оказаться, что при удалении вершины
старая компонента связности распадается сразу на три или ещё больше. Обратите также внимание, что нельзя говорить
"<мост "--- это такое ребро, при удалении которой граф распадётся на \textit{две} компоненты связности"> и аналогично для 
точек сочленения, т.к. не понятно, что это значит для несвязных графов, и в наиболее логичной трактовке для 
несвязных графов это неверно.}

\answer{bridgesandSV}{Я такой связи не знаю. а) нет, т.к. конец моста может оказаться висящей вершиной.
Вообще, например, в графе с двумя вершинами и одним ребром (1--2) один мост, но ни одной точки сочленения.
б) Очевидно, нет, и несложно придумать контрпример.}

\answer{crossedges}{Пусть есть такое ребро. Рассмотрим тот конец его, в который мы попали раньше "--- пусть это
вершина $u$. К моменту, когда мы попали в него, во второй конец ($v$) мы ещё на заходили. Мы будем
просматривать соседей вершины $u$ и наткнёмся на $v$. Если к этому моменту мы все ещё не были в $v$,
то мы в неё пойдём, $v$ станет сыном и потому потомком $u$ и ребро не будет перекрёстным. В противном случае 
мы успели побывать в вершине $v$, пока обрабатывали других соседей вершины $u$, значит, $v$ "--- потомок $u$
и ребро все равно не перекрёстное.}

\answer{dots}{Ну понятно, переменная внешнего цикла, в котором мы запускаем поиск в глубину. См. п. \ref{howtocall}.}

\answer{bridgesstupid}{Приводить алгоритм тут не буду, пишите сами :)}

\answer{bridgesadv}{Аналогично предыдущему. Мне кажется, что, если вы хорошо освоились с поиском в глубину,
то придумать и написать \textit{этот} алгоритм труда не должно составить.}

\answer{pathcover}{Если добавить $K/2$ рёбер, как это сказано в подсказке, то степени всех вершин станут чётными.
Поэтому найдём в графе эйлеров цикл и потом из него выкинем те ребра, которые мы раньше добавили. Цикл распадётся
на $K/2$ путей, которые и будут ответом на нашу задачу, т.к. меньше путей получить нельзя.

Бонусное задание: а почему именно на $K/2$? Вдруг у нас несколько удаляемых рёбер в цикле будут идти подряд?}

\answer{postman}{Задача"=шутка. Несложно доказать, что веса всех эйлеровых циклов одинаковы и равны $n(n+1)/2-S$,
где $S$ "--- сумма всех $x_i$, поэтому выводим любой цикл.}

\answer{nondirecttoacyclic}{Очень просто на самом деле, и даже поиск в глубину тут ни при чем. Просто
ориентируем все ребра от вершины с меньшим номером к вершине с б\'{о}льшим.

Вообще, любой ациклический граф, как мы знаем, можно оттопсортить, и наоборот, любой граф, который можно
оттопсортить (т.е. для которого задача топологической сортировки имеет решение), является ациклическим.
Поэтому можно просто задать произвольный порядок вершин и ориентировать ребра в соответствии с этим порядком,
и, более того, таким образом можно получить \textit{любой} ациклический граф, который вообще можно
получить из данного неориентированного.}

\answer{nondirecttoSC}{Решения не будет существовать тогда и только тогда, когда в графе есть мосты.
Если их нет, то ориентировать надо также, как при поиске мостов, т.е. ребра дерева поиска "--- от корня,
остальные ребра "--- к корню. (На самом деле даже искать мосты не обязательно).}

\answer{semidirecttoacyclic}{Выкинем все неориентированные ребра и оттопсортим полученный граф. Если это невозможно,
то задача решения не имеет (почему?). Потом вернём все неориентированные ребра и ориентируем их в соответствии
с полученным в топсорте порядком.

Бонусное задание: сравните эту задачу с задачей \ref{nondirecttoacyclic}.}

\answer{eXam}{Постройте граф: вершины "--- экзамены, ребра "--- ученики. Осталось проверить его на 
двудольность.}