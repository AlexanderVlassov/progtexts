\header{Задачи на поиск в глубину}

Конечно, я предполагаю, что вы напишите все обсуждавшиеся алгоритмы, решите все вышеприведённые задачи 
и напишите и программы по этим задачам.

\task{\label{pathcover}Дан граф. Вывести несколько путей так, чтобы по каждому ребру проходил ровно один путь и при этом число путей было 
минимальным. В каждом пути вершины могут повторяться, но ребра нет. (В частности, если в графе существует
эйлеров цикл или путь, то надо просто его вывести).}

\task{\label{postman}Дан граф, в котором точно существует эйлеров цикл (т.е. степени всех вершин чётны). Для каждого ребра $i$
дано число $x_i$. Для каждого эйлерова цикла в этом графе введём его "<штраф"> следующим образом.
Штраф за одно ребро $i$ будет равен $k-x_i$, где $k$ "--- порядковый номер этого ребра в цикле. Штраф цикла
равен сумме штрафов за все ребра. Требуется найти эйлеров цикл с минимальным штрафом.}

\task{\label{nondirecttoacyclic}Дан неориентированный граф. Ориентировать его (т.е. каждое ребро ориентировать в одно из двух возможных
направлений) так, чтобы он стал ациклическим. Если решения не существует, то сообщить это.}

\task{\label{nondirecttoSC}Дан неориентированный граф. Ориентировать его так, чтобы он стал сильносвязным. 
Если решения не существует, то сообщить это.}

\task{\label{semidirecttoacyclic}Дан частично ориентированный граф (т.е. некоторые ребра изначально ориентированные, 
некоторые нет). Ориентировать неориентированные ребра так, чтобы он стал ациклическим. Если решения не существует,
то сообщить это.}

\task{\label{eXam} (Задача 172 с acm.sgu.ru) В некоторой школе проводятся "<экзамены по выбору">. 
Каждый ученик должен выбрать два экзамена из данного списка и сдать их. Известно, какой ученик какие экзамены выбрал. 
Школа хочет выделить ровно два дня под эти экзамены, при этом в первый день провести часть 
экзаменов, а остальные экзамены "--- и только их "--- во второй. Конечно, при этом должно получиться так, 
чтобы каждый школьник мог сдать выбранные им экзамены в разные дни. Требуется составить такое 
расписание экзаменов (т.е. распределить экзамены по дням) или сообщить, что это невозможно.}