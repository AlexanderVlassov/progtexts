\header{Задачи на поиск в глубину}

Конечно, я предполагаю, что вы напишите все обсуждавшиеся алгоритмы, решите все вышеприведённые задачи 
и напишите и программы по этим задачам.

\task|Дан граф. Вывести несколько путей так, чтобы по каждому ребру проходил ровно один путь и при этом число путей было 
минимальным. В каждом пути вершины могут повторяться, но ребра нет. (В частности, если в графе существует
эйлеров цикл или путь, то надо просто его вывести).
||Ясно, что число путей будет как минимум равно половине от числа вершин с нечётной степенью; пусть
это число равно $K$, оно всегда чётно. Добавим в наш граф $K/2$ рёбер, попарно (произвольным образом)
соединив наши нечётные вершины. Что дальше?
||Если добавить $K/2$ рёбер, как это сказано в подсказке, то степени всех вершин станут чётными.
Поэтому найдём в графе эйлеров цикл и потом из него выкинем те ребра, которые мы раньше добавили. Цикл распадётся
на $K/2$ путей, которые и будут ответом на нашу задачу, т.к. меньше путей получить нельзя.

Бонусное задание: а почему именно на $K/2$? Вдруг у нас несколько удаляемых рёбер в цикле будут идти подряд?
|\label{pathcover}

\task|Дан граф, в котором точно существует эйлеров цикл (т.е. степени всех вершин чётны). Для каждого ребра $i$
дано число $x_i$. Для каждого эйлерова цикла в этом графе введём его "<штраф"> следующим образом.
Штраф за одно ребро $i$ будет равен $k-x_i$, где $k$ "--- порядковый номер этого ребра в цикле. Штраф цикла
равен сумме штрафов за все ребра. Требуется найти эйлеров цикл с минимальным штрафом.
||На самом деле это почти что задача"=шутка.
||Задача"=шутка. Несложно доказать, что веса всех эйлеровых циклов одинаковы и равны $n(n+1)/2-S$,
где $S$ "--- сумма всех $x_i$, поэтому выводим любой цикл.
|\label{postman}

\task|Дан неориентированный граф. Ориентировать его (т.е. каждое ребро ориентировать в одно из двух возможных
направлений) так, чтобы он стал ациклическим. Если решения не существует, то сообщить это.
||
||Очень просто на самом деле, и даже поиск в глубину тут ни при чем. Просто
ориентируем все ребра от вершины с меньшим номером к вершине с б\'{о}льшим.

Вообще, любой ациклический граф, как мы знаем, можно оттопсортить, и наоборот, любой граф, который можно
оттопсортить (т.е. для которого задача топологической сортировки имеет решение), является ациклическим.
Поэтому можно просто задать произвольный порядок вершин и ориентировать ребра в соответствии с этим порядком,
и, более того, таким образом можно получить \textit{любой} ациклический граф, который вообще можно
получить из данного неориентированного.
|\label{nondirecttoacyclic}

\task|Дан неориентированный граф. Ориентировать его так, чтобы он стал сильносвязным. 
Если решения не существует, то сообщить это.
||
||Решения не будет существовать тогда и только тогда, когда в графе есть мосты.
Если их нет, то ориентировать надо также, как при поиске мостов, т.е. ребра дерева поиска "--- от корня,
остальные ребра "--- к корню. (На самом деле даже искать мосты не обязательно).
|\label{nondirecttoSC}

\task|Дан частично ориентированный граф (т.е. некоторые ребра изначально ориентированные, 
некоторые нет). Ориентировать неориентированные ребра так, чтобы он стал ациклическим. Если решения не существует,
то сообщить это.
||
||Выкинем все неориентированные ребра и оттопсортим полученный граф. Если это невозможно,
то задача решения не имеет (почему?). Потом вернём все неориентированные ребра и ориентируем их в соответствии
с полученным в топсорте порядком.

Бонусное задание: сравните эту задачу с задачей \ref{nondirecttoacyclic}.
|\label{semidirecttoacyclic}

\task|(Задача 172 с acm.sgu.ru) В некоторой школе проводятся "<экзамены по выбору">. 
Каждый ученик должен выбрать два экзамена из данного списка и сдать их. Известно, какой ученик какие экзамены выбрал. 
Школа хочет выделить ровно два дня под эти экзамены, при этом в первый день провести часть 
экзаменов, а остальные экзамены "--- и только их "--- во второй. Конечно, при этом должно получиться так, 
чтобы каждый школьник мог сдать выбранные им экзамены в разные дни. Требуется составить такое 
расписание экзаменов (т.е. распределить экзамены по дням) или сообщить, что это невозможно.
||
||Постройте граф: вершины "--- экзамены, ребра "--- ученики. Осталось проверить его на 
двудольность.
|\label{eXam}