\header{Восстановление решения в задачах на ДП}
До сих пор мы обсуждали только те ситуации, когда ответ на задачу выражался одним числом 
(или boolean'ом). Но во многих задачах может быть нужно также вывести, как получить такой ответ. Например, в задаче про черепашку с поиском пути с наибольшей суммой может понадобиться не только вывести эту сумму, но и сам путь. Аналогично в задаче о наборе данной суммы с помощью данного набора монет может понадобиться вывести какой-нибудь способ набора.

В задачах на ДП в подавляющем большинстве случаев восстановление ответа делается легко, если результаты решения каждой подзадачи уже вычислены и сохранены в соответствующей матрице. Для восстановления ответа напишем процедуру $out$, которая будет выводить ответ для произвольной подзадачи. Она будет принимать в качестве параметров собственно подзадачу (например, координаты клетки в задаче про черепашку, или величины $i$ и $j$ в задаче про монеты) и выводить (в выходной файл или какой-нибудь массив) решение этой подзадачи (т.е. оптимальный путь или способ набора этой суммы; естественно, в последнем случае мы будем запускать процедуру, только если решение этой подзадачи существует). Можно то же самое сказать по"=другому: есть некоторый элемент массива (матрицы) ответов, который хранит максимальную сумму или информацию о том, можно ли набрать данную сумму, "--- процедура $out$ будет выводить подтверждение этого значения: выводить сам путь с этой суммой или сам способ набора.

\lheader{Примеры вывода решения}
Как мы это будем писать процедуру $out$? На самом деле, зная рекуррентное соотношение (я буду его называть именно так, хотя, как я уже говорил, на самом деле достаточен алгоритм, не обязательно чёткая формула), все делается элементарно. Действительно, пусть в задаче про черепашку мы хотим вывести оптимальный путь до клетки $(i,j)$. Вспоминая рекуррентное соотношение, мы знаем, что этот оптимальный путь "--- это либо путь до клетки $(i-1,j)$ и шаг вправо (считаем, что первое число в координатах "--- это номер столбца), либо путь до клетки $(i,j-1)$ и шаг вверх. Более того, зная матрицу $ans$, мы можем легко определить, какой из двух вариантов имеет место: если $ans[i-1,j]>ans[i,j-1]$, то первый, иначе второй (ну, если $ans[i-1,j]=ans[i,j-1]$, то оба). Ну тогда все просто; последняя важная идея "--- это то, что оптимальный путь до клетки $(i-1,j)$ или до $(i,j-1)$ можно вывести просто рекурсивным запуском процедуры $out$. 
\begin{codesampleo}\begin{verbatim}
procedure out(i,j);
begin
...
if ans[i-1,j]>ans[i,j-1] then begin
   out(i-1,j);
   write('R');
end else begin
   out(i,j-1);
   write('U');
end;
end;
\end{verbatim}\end{codesampleo}

Итак, ещё раз. Что делает эта процедура. Она определяет, на что заканчивается оптимальный маршрут
для клетки $(i,j)$. Если он заканчивается на шаг вправо, то этот маршрут на самом деле "--- маршрут
до клетки $(i-1,j)$, после чего идёт шаг вправо. Соответственно, процедура и поступает: она выводит
маршрут до клетки $(i-1,j)$, после чего выводит символ `R'. Аналогично второй случай.

Осталось только одно: как при вычислении массива $ans$ мы особо рассматривали некоторые случаи, так и тут в процедуре $out$ их тоже надо рассмотреть особо (если это не сделать, то будет много всего плохого, в первую очередь "--- бесконечная рекурсия). Эти особые случаи были $i=1$ или $j=1$, поэтому надо добавить соответствующие if'ы в начало процедуры:
\begin{codesampleo}\begin{verbatim}
procedure out(i,j);
begin
if (i=1)and(j=1) then exit;
if i=1 then begin
   out(i,j-1);
   write('U');
   exit;
end;
if (j=1) then begin
   out(i-1,j);
   write('R');
   exit;
end;
if ans[i-1,j]>ans[i,j-1] then begin
   out(i-1,j);
   write('R');
end else begin
   out(i,j-1);
   write('U');
end;
end;
\end{verbatim}\end{codesampleo}
(Конечно, в случае $i=1$ и $j\neq 1$ можно просто вывести строчку из нужного количества символов
`U', но мне кажется проще и более естественно и тут писать аналогично основному варианту; то же 
самое и для симметричного случая $j=1$.) (Этот код приведён для случая, когда мы не ввели нулевые элементы. Если
их ввести, то от первые if'ы резко упростятся.)

\task|Напишите процедуру $out$ для случая, когда мы ввели нулевые элементы массива так, как это 
было сказано в соответствующем разделе.
||||
\begin{codesampleo}\begin{verbatim}
procedure out(i,j);
begin
if (i=0)or(j=0) then exit;
if ans[i-1,j]>ans[i,j-1] then begin
   out(i-1,j);
   write('R');
end else begin
   out(i,j-1);
   write('U');
end;
end;
\end{verbatim}\end{codesampleo}
|\label{outzeroline}

А для задачи про монеты? Все просто и совершенно аналогично. Будет процедура $out(i,j)$, которая будет выводить способ набора суммы $j$ с помощью первых $i$ монет. Конечно, если такого способа не существует (т.е. $ans[i,j]=false$), то такой вызов бессмысленен "--- мы будем считать, что процедура $out$ всегда будет вызываться с параметрами, для которых решение существует. Тогда мы, ещё когда придумывали рекуррентное соотношение, поняли, что это решение либо включает монету $a_i$, либо не включает. Сейчас, когда мы уже насчитали всю матрицу $ans$, определить, какой из двух случаев имеет место, легко: если $j\geq a_i$ И $ans[i-1,j-a_i]=true$, то решение включает $i$-ю монету, иначе нет (на самом деле, конечно, может быть так, что возможны оба варианта, но тогда в данной задаче, ясно, все равно, какой из вариантов выбрать). Итак,
\begin{codesampleo}\begin{verbatim}
procedure out(i,j)
begin
if i=0 then exit;
if (j>=a[i])and(ans[i-1,j-a[i]]) then begin
   out(i-1,j-a[i]);
   write(i,' ');
end else 
    out(i-1,j);
end;
\end{verbatim}\end{codesampleo}
\label{coins_out}
Здесь я считаю, что мы уже ввели нулевую строку в массив $ans$, т.е. запись $ans[0,j]$ имеет смысл и поэтому именно $i=0$ является особым случаем; если бы мы это не делали, то пришлось бы особо рассматривать случай $i=1$. Обратите внимание на дополнительные достоинства введения нулевой строки: во"=первых, мы теперь рассматриваем один случай (иначе пришлось бы отдельно рассматривать случай $(i=1,j=0)$ и отдельно "--- $(i=1,j=a_1)$, т.е. писать два if'а), во"=вторых, тут в случае $i=0$ ничего не надо выводить вообще.

Обратите ещё внимание на то, как автоматически получается, что мы никогда не вызываем $out$ с параметрами, для которых нет решения. Действительно, если массив $ans$ насчитан правильно и решение для $(i,j)$ существует, то в соответствии с рекуррентной формулой оно существует или для $(i-1,j-a_i)$, или для $(i-1,j)$, причём первый вариант может иметь место только при $j\geq a_i$. Поэтому, если для $(i,j)$ решение существует, то процедура $out$ сделает рекурсивный вызов для $(i',j')$ обязательно таких, что для них решение тоже существует. Осталось нам убедиться, что вызов $out$ из главной программы выполняется только для таких $(i,j)$, для которых есть решение "--- а это наверняка так, нам же не надо вызывать $out$, если решения нет. Вообще, главная программа будет иметь вид типа
\begin{codesampleo}\begin{verbatim}
begin
считать N, S, массив a
насчитать массив ans как было показано раньше
if (ans[N,S]) then begin
   writeln('yes');
   out(N,S);
end else writeln('no');
end.
\end{verbatim}\end{codesampleo}
Естественно, если ответ `no', то $out$ мы не вызываем.

\lheader{Общая концепция написания процедуры $out$}
Итак, ещё раз общая концепция восстановления решения. Когда вы, придумывая рекуррентное соотношение,
сводите текущую подзадачу к более мелким, вы сразу автоматически понимаете, как должно выглядеть
оптимальное решение. Соответственно и процедуру $out$ вы пишете, опираясь на это понимание и
используя рекурсивный вызов для вывода ответа на ту подзадачу или те подзадачи, к которым вы свели
текущую подзадачу. Ещё раз: особенно думать при написании процедуры $out$ не надо, все, что надо 
было придумать, вы уже придумали, когда выводили рекуррентное соотношение. А теперь только 
вспомните его. Оно даёт сведение текущей подзадачи к более мелким "--- и тогда, в точности следуя 
ему, можно свести вывод решения на текущую подзадачу к выводу решения на более мелкие подзадачи, и 
применить рекурсию для вывода этих более мелких решений.

На самом деле, далеко не всегда последовательность действий в процедуре $out$ одна и та же: сначала
вызвать $out$ для подзадачи, потом вывести что-то ещё. Может быть так, что нужно вызвать $out$ для
одной подзадачи, потом что-то вывести, потом вызвать $out$ для другой подзадачи; может быть так, что
надо что-то вывести, потом вызвать $out$ для подзадачи, потом ещё что-то вывести и т.д. "--- в
каждой конкретной задаче вполне очевидно, какой именно вариант имеет место: когда вы продумываете
рекуррентное соотношение, вы сразу понимаете, как будет выглядеть соответствующее решение, "--- и
какой бы вариант ни был нужен, его очень легко реализовать в процедуре $out$.

Ещё замечу, что в рассмотренных выше примерах может возникнуть большое желание избавиться от
рекурсии, выводя ответ с конца в начало "--- это можно и довольно легко (%
\task|избавьтесь от рекурсии в какой"=нибудь из приведённых выше процедур $out$%
||||Ну, например, в задаче про монеты. Идея в том, что тут можно выводить монеты в ответ в произвольном 
порядке, в том числе и в порядке, обратном входному.
\begin{codesampleo}\begin{verbatim}
procedure out(i,j)
begin
while i<>0 do begin
      if (j>=a[i])and(ans[i-1,j-a[i]]) then begin
         write(i,' ');
         j:=j-a[i];
         dec(i);
      end else 
          dec(i);
end;
\end{verbatim}\end{codesampleo}
Или даже можно while на for заменить. Мне кажется, что такой while более аналогичен процедуре, 
которую я приводил в тексте, и лишь поэтому я не пишу его короче.

Если же монеты надо было бы выводить в правильном порядке, то можно было перед работой динамики 
перевернуть массив с монетами, чтобы такой код как раз и выводил в правильном порядке. Но, как я 
уже писал в основном тексте, имхо в большинстве случаев лучше с этим не заморачиваться.
|%
), но только мне кажется, что
рекурсивный вариант намного более прозрачен и понятен. Конечно, он использует больше памяти (на
стеке), и поэтому возможна ситуация, когда стека вам не хватит "--- тогда придётся выводить
нерекурсивно, но если все нормально и стека и времени хватает, то имхо вполне сойдёт и рекурсивная
процедура. Кроме того, далеко не в каждом из перечисленных в предыдущем абзаце вариантов можно
избавиться от рекурсии.

Единственная проблема, которая вас может ожидать при написании процедуры $out$ таким способом "---
это необходимость определять, какой именно из нескольких случаев в рекуррентном соотношении имел
место (пришли мы слева или снизу; использовали мы или нет $i$-ю монету и т.п.). Пока с этим было
просто; на самом деле, наверное, всегда можно просто ещё раз повторить вычисления, которые
проводились в рекуррентном соотношении, и тем самым все понять. Но нередко писать это лень, да ещё
дублирование кода создаст опасность лишних ошибок, наконец, повторять все проверки легко, пока у нас
всего два варианта (как и было везде выше), но их может быть больше "--- и заново перебирать их
будет лишней тратой времени. В таком случае может быть полезно при вычислении массива $ans$ сразу
запоминать, какой из случаев в рекуррентном соотношении имел место ("<откуда мы пришли в эту
клетку">), в специальном массиве $from$, и потом просто использовать его значения (если вы помните
алгоритмы поиска кратчайших путей в графе, то это все очень аналогично). Пример будет ниже.

Обратите ещё внимание, что здесь вам обычно нужно знать \textit{весь} массив $ans$, поэтому
всякие трюки с сохранением только последних строк массива не пройдут.

\lheader{Вывод лексикографически первого решения}
Иногда бывает, что при наличии нескольких решений требуют вывести какое"=нибудь определённое,
например, в некотором смысле лексикографически наименьшее. В общем случае это несколько меняет саму
задачу и это приходится учитывать в рекуррентном соотношении. Например, если бы в задаче про монеты
требовали вывести решение с наименьшим числом монет, то получилось бы задание \ref{min_coins}, над
которым, я надеюсь, вы уже подумали "--- там для соблюдения дополнительного условия приходится
привлекать "<тяжёлую артиллерию">, т.е. менять само рекуррентное соотношение, но зато в итоге вывод
ответа опять становится элементарным. Но бывают случаи, когда можно обойтись лишь простым изменением
процедуры $out$ или небольшой коррекцией основного цикла вычисления массива $ans$. Например (пример
какой"=то неестественный получается, но ничего другого в голову с ходу не приходит), пусть нам нужно
по возможность использовать монеты с большими номерами. А именно, если можно вывести решение с
монетой $a_N$, то вывести его, иначе (никуда не денешься) "--- без монеты $a_N$; среди всех таких
решений по возможности выбрать решение с $a_{N-1}$, среди всех их "--- с $a_{N-2}$ и т.д. Такой в
некотором смысле аналог лексикографического порядка. Это требование элементарно удовлетворяется; на
самом деле, если подумать, то приведённая выше процедура $out$ именно это и делает: ведь, выводя
решение $out(i,j)$, надо, если можно, вывести решение, содержащее $i$-ю монету, и только если такого
нет, то вывести решение без неё. Именно это мы и делаем. Если бы мы написали по"=другому:
\begin{codesampleo}\begin{verbatim}
procedure out(i,j)
begin
if i=0 then exit;
if (ans[i-1,j]) then 
    out(i-1,j);
else begin
   out(i-1,j-a[i]);
   write(i,' ');
end;
end;
\end{verbatim}\end{codesampleo}
т.е. по возможности не использовали бы $i$-ую монету, то выводили бы другое решение. 

Т.е. то же самое, но по"=другому: иногда в процедуре $out$ у нас возможны сразу несколько вариантов, и мы должны сделать выбор (я об этом уже говорил перед предыдущем примером процедуры $out$ для задачи про монеты). Если этот выбор сделать грамотно, то можно вывести то решение, которое надо.

Если же вы выводите решение с использованием массива $from$, то в процедуре $out$ у вас выбора нет, вы тупо следуете массиву $from$. Но тогда есть выбор при вычислении массива $from$; обычно он легко осуществляется просто правильным порядка перебора вариантов; пример будет ниже.

Ещё обратите внимание, что в наиболее простых вариантах мы выводим решение с конца, поэтому и
выбирать мы можем только так: сначала выбирать последний элемент, только выбрав его, выбирать
предпоследний и т.д. Если это именно то, что надо (например, как в рассмотренном только что варианте
с монетами), то круто, иначе (например, если бы потребовали по возможности использовать
\textit{первую} монету и т.д.), были бы проблемы. Их, наверное, в некоторых случаях можно решить,
решая задачу "<задом наперёд">; в данном случае достаточно просто обратить порядок монет, сделав
последнюю первой и наоборот. Даже более того, очень часто задача в некотором смысле симметрична, и 
потому, когда надо, можно смотреть, \textit{на что начинается} решение (и в задаче про монеты смотреть,
можно ли набрать сумму $j$ с помощью монет $a_i$, $a_{i+1}$, \dots, $a_N$, сводя эту задачу к задаче с б\'{о}льшим $i$).

\task|Научитесь выводить первое в лексикографическим порядке решение задачи про черепашку с набором 
максимальной суммы. Решение задаём строкой из букв `R' и `U' и лексикографический порядок на этих 
строках определяем как всегда.
||||Ну, во"=первых, модернизируем динамику так, чтобы можно было выводить решение с начала, а не с 
конца. Для этого для каждого $i$ и $j$ в $ans[i,j]$ будем хранить лучшую сумму от $(i,j)$ до 
$(N,M)$. Рекуррентное соотношение и основной цикл динамики напишите сами, я приведу процедуру 
$out$. Предполагаю, что мы уже ввели нулевую строку и столбец аналогично ответу \ref{outzeroline} 
(правда, тут это будет $N+1$"=ая строка и $(M+1)$"=ый столбец).

\vbox{\begin{codesample}\begin{verbatim}
procedure out(i,j);
begin
if (i=N+1)or(j=M+1) then exit;
if ans[i+1,j]>ans[i,j+1] then begin
   write('R');
   out(i+1,j);
end;
if ans[i+1,j]=ans[i,j+1] then begin
   write('R');
   out(i+1,j);
end;
if ans[i+1,j]<ans[i,j+1] then begin
   write('U');
   out(i,j+1);
end;
end;
procedure out(i,j);
begin
if (i=N+1)or(j=M+1) then exit;
if ans[i+1,j]>=ans[i,j+1] then begin
   write('R');
   out(i+1,j);
end else begin
   write('U');
   out(i,j+1);
end;
end;





\end{verbatim}
\end{codesample}}
Слева приведено простое решение, которое чётко показывает, что мы делаем: если один из двух 
имеющихся у нас вариантов явно лучше другого (т.е. $ans[i+1,j]\neq ans[i,j+1]$), то мы идём туда. 
Иначе, если оба равноценны, то надо идти туда, где первый ход будет лексикографически наименьшим, 
т.е. идти `R'. Справа "--- вариант, который показывает, что это же можно написать и проще, 
объединив варианты хода вправо "<потому что туда выгоднее"> и "<потому что все равно, куда идти">.
|\label{tortoise:firstlex}

\lheader{Нетривиальный пример: задача про наибольшую возрастающую подпоследовательность}
Дан массив $a_1$, $a_2$, \dots, $a_N$. Требуется вычеркнуть из него как можно меньше чисел так, чтобы получилась строго возрастающая последовательность. (Конечно, можно потребовать нестрого возрастающую, или убывающую "--- все будет аналогично.) Например, для массива \mbox{1 3 3 1 4 2} решением будет оставить \mbox{1 3 4}. 

Будем решать задачу динамическим программированием. На самом деле есть какой"=то довольно простой
способ решить задачу за $O(N\log N)$, если я не ошибаюсь, но мы не будем мучиться и решим её за
$O(N^2)$. Итак, для каждого $i$ посчитаем длину наибольшей возрастающей подпоследовательности куска
$a_1$, \dots, $a_i$ при условии, что $a_i$ обязательно входит в эту последовательность, и запишем
эту длину в $ans[i]$. Например, для приведённого выше примера массив $ans$ должен будет иметь вид
\mbox{1 2 2 1 3 2}. Ясно, что мы считаем?

Выбор подзадач кажется немного странным. Может захотеться реализовать ДП для тех подзадач, 
которые первыми приходят в голову: $ans[i]$ будет длиной наибольшей возрастающей последовательности 
среди первых $i$ чисел, но я не знаю, как такие подзадачи свести к более мелким. Дело в том, что 
последовательность должна быть возрастающей, а потому при сведении к более мелким подзадачам нам 
важно как"=то быть уверенными, что ответ на более мелкую подзадачу не закончится на слишком большое 
число "--- поэтому проще всего знать это самое число.
Вообще, это довольно стандартный приём "--- в 
задачах на выбор элементов потребовать, чтобы последний элемент обязательно входил.

Считать на самом деле это просто. \textit{На что может заканчиваться} такая подпоследовательность? Ну ясно дело, что на $a_i$ по условию "--- поэтому будем смотреть на число, стоящее перед ним. Ясно, что это может быть любое число, идущее до $a_i$ в начальном массиве и строго меньшее, чем $a_i$, т.е. это может быть такое $a_j$, что $1\leq j<i$ и $a_j<a_i$. Более того, ясно, что тогда началом нашей подпоследовательности будет наибольшая возрастающая подпоследовательность, заканчивающаяся на $a_j$ "--- а её длину мы уже посчитали; она равна $ans[j]$. Итак, ясно, что $ans[i]$ равен 1 плюс максимум $ans[j]$ по всем $j$ таким, что $1\leq j<i$ и $a_j<a_i$. Можно записать и явное рекуррентное выражение, но я думаю, что смысла в этом мало: понимания оно не прибавит, только потребует дополнительных размышлений на тему того, что же тут такое написано "--- т.е. этот как раз тот случай, про который я говорил выше: достаточно алгоритма вычисления $ans[i]$, а соотношение, собственно, не важно.

Да, ещё. Пока у нас особым случаем ("<базой ДП">) является случай, когда в оптимальном решении перед $a_i$ ничего не идёт "--- тогда ответ будет $ans[i]=1$. Это нужно или особо учитывать (на самом деле это совсем просто), или ввести нулевой элемент массива $ans[0]=0$ и нулевой элемент массива $a[0]=-\infty$. Несложно видеть, что это удовлетворяет основному требованию на нулевые элементы: все остальные значения тогда правильно посчитаются по алгоритму общего случая.

Итак, получаем следующий цикл:
\begin{codesampleo}\begin{verbatim}
ans[0]:=0;
a[0]:=-inf; //ну понятно, число, которое меньше любых a[i]
for i:=1 to n do begin
    max:=-1; 
    for j:=0 to i-1 do
        if (a[j]<a[i])and(ans[j]>max) then
           max:=ans[j];
    ans[i]:=max+1;
end;
\end{verbatim}\end{codesampleo}
Ещё раз. Мы вычисляем максимум $ans[j]$ по всем $j$ таким, что $0\leq j<i$ (мы ввели нулевой элемент и потому тут стоит ноль, а не единица; обратите внимание, что это условие учитывается границами цикла) и $a_j<a_i$ (а это уже приходится писать в if), и тогда $ans[i]$ на единицу больше этого максимума.

Тут немного нетривиально находится ответ на задачу: раньше у нас всегда ответ лежал в $ans[N]$ и т.п., а здесь ответ, очевидно, есть максимальное из всех значений $ans$. Но это несложно и вы это сможете сделать сами; обсудим то, ради чего я тут и даю этот пример.

Как вывести саму подпоследовательность"=решение? Ясно, мы напишем процедуру $out(i)$, которая будет выводить наибольшую возрастающую подпоследовательность, заканчивающуюся на $a_i$. Но как вспомнить то $j$, на котором мы нашли максимум? На самом деле можно в процедуре $out$ ещё раз пробежаться по всем $j$ и найти подходящий, но проще будет завести массив $from$, в $i$-м элементе которого мы и будем хранить то $j$, на котором пришёлся максимум:
\begin{codesampleo}\begin{verbatim}
ans[0]:=0;
a[0]:=-inf; //ну понятно, число, которое меньше любых a[i]
for i:=1 to n do begin
    max:=-1; 
    for j:=0 to i-1 do
        if (a[j]<a[i])and(ans[j]>max) then begin
           max:=ans[j];
           maxj:=j;
        end;
    ans[i]:=max+1;
    from[i]:=maxj;
end;
\end{verbatim}\end{codesampleo}
Обратите (здесь и далее) внимание, что элемент $from[0]$ нам не нужен.

Теперь $out$ пишется совсем элементарно:
\begin{codesampleo}\begin{verbatim}
procedure out(i)
begin
if i=0 then exit;
out(from[i]);
write(a[i],' ');
end;
\end{verbatim}\end{codesampleo}
(мы выводим сами числа, а не их номера). Обратите внимание на простоту кода: никаких вариантов, просто тупо следуем массиву $from$.

А если теперь хотим лексикографически наименьшую последовательность в ответе? Ну например так: надо вывести ответ с наименьшим последним элементом, среди всех таких "--- ответ с наименьшим предпоследним элементом и т.д. (на самом деле, по"=моему, в силу специфики данной конкретной задачи, здесь это то же самое, что и требовать наименьший первый элемент, среди всех таких наименьший второй элемент и т.д., т.е. настоящий лексикографический минимум, можете над этим подумать "--- но для простоты мы рассмотрим именно такой "<задом"=на"=перед"> лексикографический порядок). Раньше мы бы в $out$ это учли, а теперь придётся учитывать в вычислении $from$ "--- но ясно, что это просто требует при прочих равных отдавать предпочтение меньшим по значению элементам $a_j$:
\begin{codesampleo}\begin{verbatim}
...
    for j:=0 to i-1 do
        if (a[j]<a[i]) then
           if (ans[j]>max)or((ans[j]=max)and(a[j]<a[maxj])) then begin
              max:=ans[j];
              maxj:=j;
           end;
...
\end{verbatim}\end{codesampleo}
Т.е. если мы нашли ещё один вариант с такой же длиной, то иногда имеет смысл перезаписать наше старое решение.

Все, после этого будет выводиться нужное решение.

\lheader{Пример на нетривиальную процедуру $out$: алгоритм Флойда с сохранением промежуточной вершины}
Я не буду подробно рассказывать здесь алгоритм Флойда, просто скажу, что он ищет кратчайшие пути между всеми парами вершин графа, при этом в одном из вариантов в массиве $from$ в элементе $from[i,j]$ хранит некоторую вершину, лежащую на кратчайшем пути из $i$ в $j$ (или $-1$, если кратчайший путь из $i$ в $j$ не проходит ни через какие другие вершины). Тогда процедуру $out$ придётся писать так:
\begin{codesampleo}\begin{verbatim}
procedure out(i,j)
begin
if from[i,j]=-1 then begin
   write(i,' ');
   exit;
end;
out(i,from[i,j]);
out(from[i,j],j);
end;
\end{verbatim}\end{codesampleo}

Она в общем случае выводит путь от $i$ до промежуточной вершины, а потом от промежуточной вершины до $j$ "--- т.е. это пример на ту самую нетривиальность процедуры $out$, про которую я говорил как"=то выше: она теперь не делает рекурсивный вызов и потом что"=то нерекурсивно выводит, а делает что"=то более хитрое.

Ещё тут есть тонкость, что такая процедура $out$ не выводит последнюю вершину пути, иначе на стыке вершина $from[i,j]$ была бы выведена дважды, но это сейчас не принципиальный момент.
