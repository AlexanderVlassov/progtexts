\lheader{Быстрое возведение в степень и умножение} 
Пусть нам надо число $a$ возвести в степень $p$ или умножить на число $p$ (примеры кода для 
умножения и возведения буду приводить параллельно в двух колонках; число $p$ должно быть 
натуральным, а $a$ может быть произвольным). Для задачи возведения в степень будем считать, что мы умеем 
только умножать, для задачи умножения будем считать, что мы умеет только складывать (на самом деле 
задачу про умножение я привожу тут только для понятности "--- мне кажется, её понять может быть проще. В 
реальности я не представляю случая, когда вам пришлось бы так умножать). Основной текст будет про 
возведение в степень, только примеры буду приводить параллельно для обеих задач.

Конечно, можно сделать это тупо, написав что-нибудь вроде:

\begin{codesample}\begin{verbatim}
ans:=1; {в этой переменной будем хранить ответ}
for i:=1 to p do ans:=ans*a;
ans:=0; {в этой переменной будем хранить ответ}
for i:=1 to p do ans:=ans+a;
\end{verbatim}
\end{codesample}

К сожалению, этот цикл выполняется за $O(p)$. Если $p$ маленькое, то можно (и нужно!) не мучиться и написать этот простой цикл 
(ведь любое усложнение программы повышает вероятность ошибки). Если же $p$ большое, или же $p$ 
"<среднее">, но возводить в степень придётся много раз, то лучше делать все за $O(\log p)$ (ясно, что, если $p$ совсем маленькое, то в 
любом случае надо писать по"=тупому). Этот метод на самом деле интуитивно понятен. Если вам нужно посчитать $6^8$, то можно просто $6$ 
возвести в квадрат, потом ещё раз в квадрат, потом ещё раз в квадрат. Эту идею мы будем использовать в нашем алгоритме.

Пусть нам надо посчитать $a^{43}$. Разложим $43$ по степеням двойки: $43 = 32 + 8 + 2 + 1$. Тогда 
$a^{43} \hm= a^1 \cdot a^2 \cdot a^8 \cdot a^{32}$. Будем в переменной $t$ считать $a^1$, потом $a^2$, 
$a^4$, $a^8$, $a^{16}$ и т.д. (просто последовательно возводя $t$ в квадрат) и постепенно 
умножать на некоторые из них (на те, которые нам нужны) текущий результат. Как проверить, 
какие степени нам нужны? Очень просто, см. в части II текущей темы :) про побитовые операции. 
Степень $a^{2^i}$ нам нужна тогда и только тогда, когда $p \AND 2^i \neq 0$.

{\small Аналогично для умножения: например, $43a=32a+8a+2a+1a$, в переменной $t$ будем 
последовательно считать $1a$, $2a$, $4a$, $8a$ и т.д. путём последовательного сложения $t$ с собой; 
необходимые числа будем добавлять к ответу.

}

\begin{codesample}\begin{verbatim}
ans:=1;
t:=a;
cp:=1; {текущая степень, т.е. t=a^(2^cp)}
while 1 shl cp<p do begin
  if p and (1 shl cp)<>0 then ans:=ans*t;
  t:=t*t;
  inc(cp);
end;
ans:=0;
t:=a;
cp:=1; {t=a*(2^cp)}
while 1 shl cp<p do begin
  if p and (1 shl cp)<>0 then ans:=ans+t;
  t:=t+t;
  inc(cp);
end;
\end{verbatim}
\end{codesample}
(Немного про условие $1 \SHL cp<p$: ясно, что нам надо действовать до максимальной степени двойки, 
входящей в $p$, а это как раз и даёт такое условие)

Можно фактически эту же идею реализовать и по"=другому. 
Посмотрим на последнюю цифру числа $p$ в двоичной записи (т.е. на остаток от деления $p$ на 2), 
и, если это 1, то умножим текущий результат на $a$. Отбросим последнюю цифру $p$
(просто поделим на 2, сохранив только целую часть). Посмотрим опять на последнюю цифру (она раньше была второй). 
Если это 1, то нужно умножить на $a^2$. 
Опять отбросим последнюю цифру и посмотрим на ту, что теперь стала последней. Если она 1, то 
ответ надо умножить на $a^4$ и т.д., пока число $p$ не кончится. В отличие от предыдущего способа, 
мы не будем считать текущую степень отдельно, а будем менять $p$, постепенно деля его пополам, так, 
чтобы нужная нам цифра всякий раз оказывалась последней.
Вот код:

\begin{codesample}\begin{verbatim}
ans:=1;
t:=a;
while p>0 do begin
  if p and 1 = 1 then ans:=ans*t;
  p:=p shr 1;
  t:=t*t;
end;
ans:=1;
t:=a;
while p>0 do begin
  if p and 1 = 1 then ans:=ans+t;
  p:=p shr 1;
  t:=t+t;
end;
\end{verbatim}
\end{codesample}

Код совсем несложный, главное "--- чётко понимать, что вы делаете и что тут происходит :) 
Понять корректность этого кода можно и следующим образом. Заметим, что в каждый 
момент у нас будет верно, что $(ans\cdot t^p)$ равно требуемому ответу (так называемый 
"<инвариант цикла">). Действительно, изначально $ans=1$, 
$t=a$, а $p$ ещё не менялось, поэтому $ans\cdot t^p=t^p$ как раз и есть то, что мы и хотим 
получить. Далее, на очередном шаге есть два варианта: если $p$ чётное, то 
$ans\cdot t^p=ans\cdot (t^2)^{(p/2)}$, т.е. можно просто поделить $p$ на 2 и возвести $t$ в квадрат, при 
этом инвариант не нарушится. Если же $p$ нечётное, то "<отщепим"> от $p$ единицу: 
$ans\cdot t^p=(ans\cdot t)\cdot (t^2)^{(p/2)}$ (деление пополам тут имеется ввиду в смысле 
div~2, или shr~1, т.е. с сохранением только целой части). В конце концов, когда $p$ станет равно 0, получится, что 
искомый ответ равен $ans\cdot t^p=ans\cdot t^0=ans$, т.е. он уже лежит в переменной $ans$. 
Фактически, мы как бы "<перелили"> множители из $t^p$ в $ans$.

Нетрудно видеть, что цикл работает за $O(\log p)$, т.к. каждый раз число $p$ уменьшается хотя бы 
в два раза (ровно в два раза, когда на конце нолик, и чуть больше, чем в два раза, когда единичка).

Дальнейшие рассуждения имеют смысл в первую очередь именно для возведения в степень, а не умножения. 
Как уже было сказано выше, данный код имеет смысл писать, когда число $p$ большое. Но тогда $a^p$, 
скорее всего, не влезет ни в какую память, даже в extended, int64 и т.д. Конечно, может стоять 
задача вычисления степени длинной арифметикой, но тут "--- внимание! "--- приведённый алгоритм 
не столь эффективен, т.к. требует умножения \textit{длинного на длинное}, в то время как тупой 
алгоритм требует лишь умножения \textit{длинного на короткое} (можете попробовать оценить сложности 
обоих алгоритмов, учитывая, что в ответе будет $O(p)$ цифр; у меня получилась сложность $O(p^2)$ 
для тупого алгоритма и $O(p^2\log p)$ для "<продвинутого">, т.е. "<продвинутый"> в этом случае "--- 
при использовании длинной арифметики "--- даже хуже простого).

Но нередко бывает нужно вычислить ответ только по некоторому (большому, но лезущему в longint и т.п.) модулю $inf$, т.е. 
вычислить $a^p \bmod inf$ (от слова infinity "--- бесконечность:) ).  Это, пожалуй, и есть как раз 
самый основной случай применения этого алгоритма.

\begin{codesampleo}\begin{verbatim}
ans:=1;
t:=a;
while p>0 do begin
  if p and 1 = 1 then ans:=(ans*t) mod inf;
  p:=p shr 1;
  t:=(t*t) mod inf;
end;
\end{verbatim}
\end{codesampleo}

Тут, конечно, надо быть осторожным: если хранить все переменные в лонгинте, а $inf\sim 10^9$ (т.е. 
порядка $10^9$), то во время умножения $ans*t$ или $t*t$ может произойти переполнение; надо 
подумать, что с ним делать.

Ну и ещё один комментарий: это как раз тот случай, когда надо писать shr~1, а не div~2, как и 
написано везде выше.
