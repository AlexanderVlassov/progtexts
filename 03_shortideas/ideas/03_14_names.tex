\lheader{О названиях переменных, отступах и т.д.} 
На самом деле это отдельный сложный вопрос, и большинство в этой теме "--- это вопрос привычки и вкуса. Поэтому не смотрите на это как на обязательные требования, а просто как на какие-то идеи "--- вдруг что полезное для себя узнаете.

Текст написан сначала Оксаной Побуринной (О.П.), потом потом подредактирован мною (П.К.) Большие примечания я писал в скобках, 
но и делал мелкие правки без указания, что это я поправил. Если в других частях я мог 
редактировать текст так, чтобы не приходилось явно писать, где что я добавил, то тут не так, 
поскольку очень многое тут есть вопрос вкуса. 

У каждого человека свой стиль написания программы, свои названия переменных, отступы и т.д. Но, во"=первых, 
нужно, чтобы этот стиль почти не менялся от программы к программе, а во"=вторых, есть неписаные общепринятые 
правила, которых лучше придерживаться. 

\textbf{1.} По поводу названий переменных.

 Во"=первых, полезно привыкнуть в разных программах одно и 
то же называть одними и теми же 
именами. Т.е. если вы привыкли, что $i$, $j$, $k$ "--- это переменные для циклов, $res$ или $ans$ "--- это ответ на задачу,
$cur$, $t$ "--- это некое текущее значение, то по возможности так везде и пишите. Это вам поможет, 
чтобы 1. не думать при написании новой программы, как какую переменную назвать; 2. когда вы глядите на 
программу, вам будет намного проще вспомнить, что какая переменная обозначает.

Во"=вторых, называйте и используйте переменные так, чтобы вам в них не запутаться. Если массивов 
много, не надо называть их $a$, $b$, $c$, $d$, $e$ и $f$. Не  
поленитесь и напишите $cost$, $color$, $was$ и т.д. Конечно, придумывать длинные названия тоже не 
надо, надо почувствовать баланс. Если у вас всего один массив, то напишите $a$ и не беспокойтесь. 
Если два, то не нужно придумывать $IsVertexVisited$ или $VertexLastVisitedTime$, если хватит просто 
$vis$ и $ltime$. Если у вас огромное множество переменных, то, конечно, никуда от фраз вида
\begin{codesampleo}\begin{verbatim}
if (Client[i].status=_free)and(not IsPinging(Client[i].sock)) then begin
или
pALLshutDownAnswer=^tALLshutDownAnswer;
\end{verbatim}
\end{codesampleo}
\noindent не денешься. Но это в больших приложениях, в олимпиадных задачах такого, как правило, не бывает. (полабзаца 
добавлены мною "--- П.К.)

Если у вас все"=таки 4 массива $a$, $b$, $c$, $d$, 
и вы очень хорошо помните, что $n$ "--- размер массива $a$, $m$ "--- размер $b$, $k$ "--- размер 
$c$ и $l$ "--- размер $d$, то пожалуйста, пишите. Только ещё раз подумайте, действительно ли вы так 
хорошо все помните. А лучше как минимум вместо $n$, $m$, $k$ и $l$ написать $na$, $nb$, $nc$ и $nd$ (хотя мне 
это тоже не очень нравится) (\textsl{Да, мне тоже. Потому что нечего называть массивы $a$ \dots{} 
$d$. Назовите массивы по их смыслам "--- хотя бы $w$ если это вес, $name$ если это имя и т.п. "--- 
и тогда пишите хотя бы $nw$, $nname$ и т.п.. Хотя все равно нечасто бывают массивы разной длины в одной программе. "--- 
П.К.})

  Ещё полезно завести себе переменную для обмена "--- такую, которая вряд ли понадобится где-нибудь ещё 
(я лично всегда пишу $ex$ и не думаю, какие из переменных $a$, $b$, $c$, $aa$, $ii$, $jj$, $s$, 
$p$, $q$, $w$ мне не жалко для строчки $ex:=a[i]; a[i]:=a[j]; a[j]:=ex;$ "--- \textsl{А я редко 
пишу обмен где-нибудь, кроме как внутри процедур сортировки, в этих процедурах не так много 
переменных, но все равно завожу специальную $t$ "--- П.К.})
Если в коде много обменов разных типов, то можно и так сделать: вместо \texttt{var s:array [1..100] 
of string;} написать
\begin{codesampleo}\begin{verbatim}
var s:array [0..100] of string;
...
s[0]:=s[i]; s[i]:=s[j]; s[j]:=s[0];
\end{verbatim}
\end{codesampleo}
Т.е. завести дополнительный элемент в массиве и использовать его для обмена. Это гораздо приятнее, чем заводить 
5 новых переменных ради одной"=единственной строчки в коде. \textsl{(Вообще, не жадничайте и не используйте одну и туже переменную в существенно разных целях "--- если у вас нет проблем с памятью, конечно. Если вам нужна временная переменная в отдельном куске программы, не думайте, какая из переменных $i$, $j$ и т.д. у вас не занята тут, а выделите новую переменную. "--- П.К.)}

Также не помешает использовать одни и те же имена для массивов в алгоритмах на графах. В частности, для поисков в глубину и 
ширину "--- хранить, были мы в вершине или нет: например, логические массивы $vis[i]$, 
$visited[i]$, $was[i]$ и т.п. Тут тоже есть проблема "--- часто нужно больше, чем 2 значения был/не 
был: например, в поиске в глубину и ориентированном графе или в поиске в ширину, если нам нужно 
помнить, за сколько шагов мы дошли до вершины и т.п. Тут можно завести массив $c$ (если он больше нигде не 
нужен), $col$ или $color$ для поиска в глубину, массив $d$, $dist$ расстояний для поиска в ширину 
(\textsl{Я не люблю логические массивы: слова true и false больно уж длинные "--- и потому даже в 
случае, когда значений надо всего два, использую массив целых чисел. И пишу в условии 
\texttt{if was[i]<>0}. "--- П.К.})

  Ещё придумайте себе какое-нибудь имя для бесконечности и тоже пишите его одинаково: $INF$, $inf$, $infinity$. 
(\textsl{Ну, конечно, если надо для неё особое имя. Например, в качестве вещественной бесконечности 
я обычно использую $1e20$, и не морочусь с заведением константы для него. Не очень правильно (а 
вдруг придётся её изменять), но эффективно. "--- П.К.})

  Ну и ещё немного. Если раньше написать процедуры nod, nok и poisk было нормально, то сейчас так не модно :) 
Обычно все-таки пишут по-английски, а не транслитом.

Ещё замечание от меня "--- П.К. Если замечаете, что часто перепутываете, как назвать ту или иную 
переменную, выберите один вариант и возьмите в привычку всегда его придерживаться. Два примера: первое: как 
называть массив, например, имён: name или names? Хорошо звучит \texttt{var names:array... of string}: <<\textit{имена "--- 
массив строк}>>. Но потом \texttt{names[i]:=} звучит коряво: <<\textit{имена $i$-ое присвоить}>>. Поэтому 
я для себя решил, что всегда в таких случаях пишу без конечного $s$. Ещё пример: как назвать 
наибольшую координату: $maxx$ или $xmax$? Я так и не решил для себя, а зря. Неприятно бывает, когда 
написал программу и первый раз компилишь её, обнаружить, что в половине мест написал $maxx$, а в половине 
$xmax$\dots

\textbf{2.} Про отступы. Стиль, конечно, у каждого свой, но писать без отступов точно не надо. Это крайне затрудняет чтение 
кода и поиск ошибок. На сколько делать отступ "--- решать вам. Но имхо лучше делать постоянный, например, всегда 1, 
всегда 2 и т.д., а табуляция в паскале таким постоянством не отличается. (\textsl{А мне наоборот 
паскалевский стиль отступов нравится больше "--- П.К.})
Отступ размера 1, мне кажется, маловат: пусть у нас много вложенных циклов
\begin{codesampleo}\begin{verbatim}
for i:=1 to n do begin
 for j:=1 to m do begin
  for ... begin
   //тут много строк кода
  end;
  //тут много строк кода
 end;
 //тут много строк кода
end;    
\end{verbatim}
\end{codesampleo}
Когда begin и end далеко друг от друга, то очень плохо видно, какой end какому begin соответствует. (В принципе, 
это же иногда верно для отступа 2, но в гораздо меньшей степени. Я обычно делаю отступ 2, и меня вполне 
устраивает). Имхо, нормальные отступы "--- 2 или 3. 4 уже много "--- 3 вложенных цикла сильно сдвигают код вправо, 
что опять же неудобно (особенно если в борланде писать, а не в фаре "--- в фаре"=то терпимо ещё более"=менее). 
Опять же надо сказать, что размер отступа зависит от языка программирования. (Например если писать на java 
где-нибудь на работе, когда нельзя переменные называть $a$, $b$, $c$, а нужно писать так, чтобы было понятно, что они 
обозначают, то на фоне функций типа \texttt{Database.Fields.NumberOfDocuments.toString} отступ на 2 символа вообще 
незаметен :) )

  Опять же непонятно, как и где именно делать отступ. Вот примеры:
  
\begin{codesample}\begin{verbatim}
for i:=1 to n do 
begin
  //чего-нибудь
  //
end;

for i:=1 to n do 
  begin
      //чего-нибудь
      //
      end;

for i:=1 to n do 
  begin
      //чего-нибудь
end; (но это вроде совсем изврат:) так писать не стоит)
(собственно, предыдущее тоже изврат имхо --- П.К.)

for i:=1 to n do begin
  //чего-нибудь
end; (это как пишу я-О.П. и я-П.К. :) :)
\end{verbatim}
\end{codesample}

Как последний вариант, тоже писать нехорошо (\textsl{а имхо самый нормальный способ "--- П.К.}), но просто мне 
не нравится писать begin в отдельной строчке "--- когда много циклов, это сильно удлиняет прогу, и 
на экран влезает гораздо меньше.

\textbf{3.} Кстати, как правильно где-то было сказано, \textit{однотипность} отступов, имён и т.д., намного важнее чем сами отступы и т.д. Т.е. если вы редактируете чужую программу, и в ней отступы/пробелы сделаны не так, как вам нравится, то лучше оставить так, как там сделано, и свой код писать в соответствии с теми отступами. И, конечно, если вы пишите свою программу, то тоже стоит все делать однотипно.