% Исходный LaTeX-код (c) Пётр Калинин
% Код распространяется по лицензии GNU GPL (!)

\lheader{Битовые операции} Всем известны логические операции and, or, xor, not. Они из двух 
boolean"=значений делают одно: $true \OR false = true$ и т.п.. Существуют аналогичные операции
для \textit{чисел}, которые выполняют соответствующие действия побитово. В паскале эти действия
записываются теми же операторами, что и логические. Таким образом, например, $3 \OR 5= 7$, так как
$3=\dots00011_2$, а $5=\dots00101_2$, поэтому, если про-or-ить побитово (т.е. столбиком, т.е. 
младший бит с младшим и т.д.), то 
получится $\dots00111_2=7$. Аналогично $3 \AND 5 = 1$ и $3 \XOR 5 = 6$ (xor "--- исключающее ИЛИ, 
т.е. равно 1, только если \textit{ровно} один бит из двух равен единице, или, что то же самое, если 
два бита различны).

С побитовым not немного хитрее: т.к. \textit{каждый} бит инвертируется, то результат зависит от 
того, сколько всего бит было в числе, т.е. от типа, в котором вы проводите вычисления. Например, в 
byte будет, видимо, $\NOT 5= 250$.

\task|Поэкспериментируйте с операцией not в других типах. В частности, верно ли, что в 
знаковых типах (shortint, integer, longint) not может давать отрицательные значения? Должен бы.|||||

Кроме этих операций, есть ещё две довольно полезных: shr и shl. Операция shr (shift-right) осуществляет 
сдвиг битовой записи "<вправо">\footnote{Кавычки потому, что не очень"=то ясно, где здесь право, а 
где лево :). Младший разряд "--- это правый или левый?} на указанное количество бит, при этом 
младшие биты отбрасываются (т.е. операция эквивалентна  
делению на степень двойки): $2=4 \SHR 1=8 \SHR 2=16\SHR 3=32\SHR 4$ (т.е. $32 \SHR 4$ "--- это 32 
сдвинуть на 4 бита вправо). Кроме того, $2=5\SHR 1=11\SHR 2=18\SHR 3=40\SHR 4$, и $5\SHR 4=0$, т.к. младшие биты 
отбрасываются. Операция shl сдвигает битовую запись влево, дополняя младшие разряды нулями (т.е. 
эквивалентно умножению на степень двойки). Старшие 
разряды, вылезающие за тип, видимо, отбрасываются. Результат, конечно же, зависит от типа. Пример: 
в любом типе $2\SHL 1=4$, $5\SHL 4=80$; если влезет в тип, то $1\SHL k=2^k$.

\task|Поэкспериментируйте с операцией shl в других типах. В частности, верно ли, что в 
знаковых типах (shortint, integer, longint) shl может давать отрицательные значения? Вроде действительно может давать.|||||

Огромное достоинство битовых операций состоит в том, что они выполняются \textit{очень} быстро (см.
также ниже, в параграфе про скорость операций). Например, делить на два с помощью shr~1 намного 
быстрее, чем с помощью div 2. Это, правда, не всегда важно "--- если операций деления на два у вас 
не так много, то сильно программу замена div~2 на shr~1 не ускорит, но иногда может помочь (см. также там же ниже). А вот возведение 
двойки в степень писать как $1\SHL k$ намного проще, чем циклом, и работать будет быстрее. Только 
ВНИМАНИЕ: стандартная ошибка "--- написать тут 2 вместо 1: например для деления на 2 написать 
shr~2 по аналогии с div~2 :) вместо верного shr~1, аналогично можно случайно написать $2\SHL k$ для 
возведения 2 в степень $k$. Это, конечно, приведёт к неправильному результату.

Битовые операции вы применяете, обычно, тогда, когда вам действительно надо что"=то сделать, 
связанное с битами. Например, число из $k$ единиц ($11\dots1_2$) можно получить как $1\SHL k-1$. 
Входит ли $k$"=я степень двойки в двоичное представление числа $n$ можно проверить так: $(n \SHR k) 
\AND 1=1$ или "--- второй способ "--- $n \AND (1 \SHL k)<>0$. В частности, проверить, чётно или нечётно число можно, 
посмотрев, чему равно $n \AND 1$, и т.п.

Обратите внимание, что я везде, где надо, ставлю скобки. Т.к. запомнить порядок действий здесь я не 
могу, то лучше для однозначности ставить скобки.

\task|Как быстро с помощью битовых операций вычислить максимальную степень двойки, на 
которую делится данное число? Ответ должен быть просто арифметическим выражением, без всяких циклов 
и т.п. и, например, для числа 40 давать ответ 8.
||Пусть нам дано число $N$. Посмотрите, чем $N-1$ отличается от $N$ в битовой записи.
||Пусть $N$ оканчивается на $k$ нулей, перед которыми идёт единица. Тогда ответ на наш вопрос будет $100\dots0_2$ "--- единица с $k$ нулями на конце. Как это вычислить? Заметим, что $N-1$ заканчивается на $k-1$ единицу, перед которыми идёт ноль (просто вычтите единицу из $N$ столбиком), т.е. отличается от $N$ ровно в $k+1$ последних битах. Тогда $N \XOR (N-1)$ будет равно $11\dots1_2$ "--- число, состоящее из $(k+1)$ единицы, "--- и $((N \XOR (N-1)) +1) \SHR 1$ даст то, что нам надо. Вообще, идея про-xor-ить $N$ и $N-1$ для выделения нулей на конце числа $N$ довольно часто встречается.
|

Ещё частое употребление "--- если вы числом $n$ кодируете последовательность нулей и единиц 
(закрашенных и незакрашенных клеток, используемых и не используемых объектов и т.п.), и вам надо 
что"=то с ней сделать (определить, закрашена ли та или иная клетка), то почти всегда подобные действия можно перевести на язык битовых операций.