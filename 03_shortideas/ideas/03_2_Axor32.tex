% Исходный LaTeX-код (c) Пётр Калинин
% Код распространяется по лицензии GNU GPL (!)

\lheader{\texttt{'A' xor ' '='a'}}
Ещё одно замечание, имеющее непосредственное отношение к кодировкам, но не к основным идеям 
предыдущего параграфа. Обратите внимание, как расположены английские маленькие и заглавные буквы в 
таблицах кодировок, приведённых в предыдущем параграфе. Они расположены друг под другом. Этот 
обозначает, во"=первых, что их коды отличаются на 32, но ещё и, во"=вторых, что у всех заглавных 
букв в разряде, соответствующем степени $2^5=32$, в двоичной записи кода стоит 0, а у маленьких "--- 1. Учитывая,
что 32 "--- это код пробела, это обозначает, что \texttt{'A' xor ' '='a'} (на самом деле здесь, 
конечно, имеются ввиду коды букв, т.е. \verb|ord('A') xor ord(' ')=ord('a')|; паскаль вам ругнётся 
на предыдущей строчек. Если вы не понимаете, что значит тут операция \verb|xor|, то посмотрите один 
из далее идущих параграфов). Это замечание, конечно, позволяет очень быстро инвертировать регистр у 
английского текста, но все равно не даёт широких возможностей для практического применения: даже 
для того, чтобы инвертировать регистр у текста, который может содержать, помимо английских букв, 
ещё и другие символы, надо каждый символ проверять, является ли он английской буквой, что сводит на 
нет всю красоту инвертации регистра таким способом; аналогичные проблемы будут, если надо, 
например, перевести весь текст в верхний регистр.

\note{Если вам надо перевести текст в верхний регистр, то лучше воспользуйтесь встроенными функциями,
либо просто прибавляйте/вычитайте 32; тем не менее имхо сам факт красив.}
