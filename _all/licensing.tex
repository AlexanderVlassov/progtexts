% Исходный LaTeX-код (c) Пётр Калинин
% Код распространяется по лицензии GNU GPL (!)

\headernono{О лицензии на эти заметки}
\addcontentsline{toc}{section}{О лицензии на эти заметки}

\small

\epigraph{...право формулировать задачу и объяснять ее решение является неотчуждаемым естественным правом всякого,\\ кто на это способен.}{А. Шень. Программирование: теоремы и задачи}

Эти заметки вы можете бесплатно скачать на сайте \verb`https://github.com/petr-kalinin/progtexts` и использовать любыми законными способами. Я не беру денег за их использование и распространение, но, соответственно, \textit{требую}, чтобы и вы не ставили никаких ограничений на использование этого текста, если вы его куда-то дальше распространяете. Более того, с того же сайте вы можете свободно скачать и исходный код этих заметок для системы \LaTeX, и вы можете в него вносить изменения и/или использовать в своих работах, но я требую, чтобы в таком случае вы сделали свободным и исходный код ваших исправлений или тех работ, где вы используете текст этих заметок.

А именно, я распространяю эти заметки на условиях лицензии GNU General Public License версии 3 (или, по вашему выбору, более старшей версии). Строгий текст лицензии вы можете прочитать на сайте Free Sofrware Foundation по адресу \verb`http://www.gnu.org/licenses/gpl-3.0.en.html` (неофициальный русский перевод: \verb`http://rus`\-\verb`gpl.ru/rusgpl.html`), или в файле \verb`COPYING.txt`, распространяемым вместе с \LaTeX-кодами этих заметок. Ниже я вкратце объясняю, что это обозначает; если вы знаете GNU GPL, то можете пропустить эти объяснения до раздела <<Дополнительные замечания>>. В частности, если вам интересно, почему я использую именно GNU GPL, а не CC BY-SA или другую CC лицензию, то читайте раздел <<Дополнительные замечания>>.

Итак, вы можете свободно использовать эти заметки при условии, что вы их не изменяете (в частности, сохраняете указание на мое авторство и указание на лицензию). Вы можете их также распространять (выставлять на сайте, распространять в печатном виде и т.п.) куда угодно на этих же условиях (т.е. по этой же лицензии). Кроме того, вы можете модифицировать эти заметки, а также использовать текст заметок в своих работах, на следующих условиях. Во-первых, вы должны распространять полученный текст (измененную версию этих заметок или свою собственную работу) также на условиях GNU GPL, в том числе не ограничивая дальнейшее распространение. Во-вторых, вы должны сделать свободно доступным на условиях GNU GPL <<\textit{исходный код}>> измененных заметок или вашей работы, т.е. тот формат, в котором вы сами вносили изменения и/или создавали свою работу. А именно, если вы правите исходный код \LaTeX, то вы должны сделать исправленный исходный код \LaTeX{} свободно доступным. Вы не можете распространять только полученный PDF или вообще только печатную версию, а исходный \LaTeX-код сделать закрытым. Или, если вы, к примеру, используете Photoshop для правки уже сгенерированного PDF, то вы должны распространять также и <<сырой>> файл PSD. Этот <<\textit{исходный код}>> вы должны или распространять сразу вместе с окончательным вариантом (PDF, печатной версией и т.п.), или в PDF или печатную версию должна быть включена информация о том, как получить этот <<\textit{исходный код}>>. Более строгие определения см. в полном тексте лицензии, ссылки на который приведены выше.

\subsection*{Дополнительные замечания}
\small

Вообще, часто для подобных <<творческих работ>> используются лицензии Creative Commons (CC). Но я распространяю эти заметки на условиях именно GNU GPL потому, что CC-лицензии не содержат понятия исходного кода и не требуют его раскрытия. Если бы я распространял заметки на условиях лицензий CC, то кто угодно мог бы внести изменения в \LaTeX-исходники, скомпилировать их в PDF и дальше распространять только PDF, а \LaTeX-исходники засекретить. Это довольно непродуктивно, т.к. скомпилированный PDF править намного сложнее, чем \LaTeX-исходники. Я хочу, чтобы все измененные версии этих заметок сопровождались исходниками, чтобы их мог править кто угодно, да и чтобы я сам мог подходящие изменения включить в оригинальный текст заметок. Соответственно, GNU GPL имеет понятие \textit{исходного кода}, причем с достаточно широким определением: <<the preferred form of the work for making modifications to it>> (<<предпочитаемая форма произведения для создания его модификаций>>). Для заметок, набранных в \LaTeX, исходный код "--- это очевидно исходный код \LaTeX, поэтому GPL позволяет достичь именно той цели, которую я сформулировал выше: она требует, чтобы всякий, кто будет распространять измененную версию заметок, распространял бы и соответствующий исходный код \LaTeX.

Кстати, исходный код этих заметок лежит в репозитории на GitHub. Если уж вы изменяете заметки, я буду очень рад, если вы пришлете pull request, чтобы я мог при желании легко включить ваши изменения в мою версию.

Еще обратите внимание, что лицензия, конечно, относится только к конкретному тексту этих заметок. Авторское право вообще защищает не идеи, а конкретные воплощения этих идей, поэтому, конечно, я не претендую и не могу претендовать на какие-либо авторские права в отношении алгоритмов, задач и т.п., описанных в этих текстах. Авторские права распространяются только на конкретную текстовую формулировку этих алгоритмов и задач.
Кроме того, конечно, я разрешаю свободное использование без каких-либо требований лицензии разумно небольших фрагментов этих заметок. Например, вы, конечно, можете использовать фрагменты кода (но не целые программы), приведенные в этих заметках, в своих программах без каких-либо ссылок на меня, или формулировки отдельных задач и т.п.